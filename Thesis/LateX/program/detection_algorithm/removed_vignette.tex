The Removal of the Vignette and lens smudges are necessary as uneven lighting impacts the performance as well as the local contrasts of the flakes, as we are taking the mean contrast of the image.\\
\\
The Vignette of the raw image only manifests in the V Channel of the HSV Colorspace, this makes it easy to remove the vignette if the the image by dividing the the V-Value by the V-Value of the flatfield image according to:

\begin{equation}
    I^V_{Corrected} = \frac{I^V_{Original}}{I^V_{Flatfield}} * \bar{I}^V_{Flatfield}
\end{equation}
\\
With $I^V$ being the intensity of the V-Channel of each pixel.\\
\\
This yields an image without Vignette and smudges on the lens. This can be quantified by comparing histograms and comparing the width of the Maxima.

\begin{figure}[ht] 
  \label{ fig7} 
  \begin{minipage}[b]{0.45\linewidth}
    \centering
    \includegraphics[width=.9\linewidth]{example-image-duck} 
    \caption{Original Image} 
    \vspace{4ex}
  \end{minipage}%%
  \begin{minipage}[b]{0.45\linewidth}
    \centering
    \includegraphics[width=.9\linewidth]{example-image-duck} 
    \caption{Color Histogram} 
    \vspace{4ex}
  \end{minipage} 
  \begin{minipage}[b]{0.45\linewidth}
    \centering
    \includegraphics[width=.9\linewidth]{example-image-duck} 
    \caption{Without Vignette} 
    \vspace{4ex}
  \end{minipage}%% 
  \begin{minipage}[b]{0.45\linewidth}
    \centering
    \includegraphics[width=.9\linewidth]{example-image-duck} 
    \caption{Histogram} 
    \vspace{4ex}
  \end{minipage} 
\end{figure}