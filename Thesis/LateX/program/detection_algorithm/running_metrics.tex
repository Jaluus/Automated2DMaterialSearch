After detecting a possible flake we can run heavy metrics on the possible hit. The Processing intensity of the metric itself is not very important, as by pre-selecting a small subset of images which contain possible flake we can minimize the impact of these metric on the run-time of the algorithm.
\paragraph{Entropy}
The entropy of the flake is determined by calculating the Shannon Entropy of the flake region. This yields a good approximation of the cleanness of the flake and is is good indicator of possible impurities. This metric is also a good indicator if the masked region is a flake, dirt on the chip or tape residues.\\
\begin{figure}[h!]
\centering
\begin{minipage}{.45\textwidth}
  \centering
  \includegraphics[width=.9\linewidth]{example-image-duck}
  \caption{Flake Closeup}
\end{minipage}
\begin{minipage}{.45\textwidth}

  \centering
  \includegraphics[width=.9\linewidth]{example-image-duck}
    \caption{The Entropy of the Flake}
\end{minipage}
\end{figure}

\paragraph{Proximity}
The proximity metric of the flake is an indicator of the vicinity of the flake, its defined as the standard deviation of pixel values in a predefined area around the flake and returns a value based on the surroundings. Flakes which are close to, or stuck to other bigger flakes are not desirable for further use.\\
\begin{figure}[h!]
\centering
\begin{minipage}{.45\textwidth}
  \centering
  \includegraphics[width=.9\linewidth]{example-image-duck}
  \caption{Flake Closeup}
\end{minipage}
\begin{minipage}{.45\textwidth}

  \centering
  \includegraphics[width=.9\linewidth]{example-image-duck}
    \caption{Proximity Area}
\end{minipage}
\end{figure}

\paragraph{Aspect Ratio}
The Aspect Ratio of the flake is being approximated by fitting a rotated rectangle to the flake mask, this yields a good approximation of the flake width and length. By dividing these values we get the Aspect Ratio.
\begin{figure}[h!]
\centering
\begin{minipage}{.45\textwidth}
  \centering
  \includegraphics[width=.9\linewidth]{example-image-duck}
  \caption{Flake Closeup}
\end{minipage}
\begin{minipage}{.45\textwidth}

  \centering
  \includegraphics[width=.9\linewidth]{example-image-duck}
    \caption{Rotated Bounding box}
\end{minipage}
\end{figure}

\paragraph{Size}
The size is a simple metric, it is easily calculated by multiplying the number of pixels of the mask by the size of a single pixel.

